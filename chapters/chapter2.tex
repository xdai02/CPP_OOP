\chapter{封装}

\section{面向过程与面向对象}

\subsection{面向过程(Procedure Oriented)}

面向过程是一种以过程为中心的编程思想,以什么正在发生为主要目标进行编程,分析出解决问题所需要的步骤,然后用函数把这些步骤一步一步实现,使用的时候一个一个依次调用。\\

C语言就是一种面向过程的编程语言,但是面向过程的缺陷是数据和函数并不完全独立,使用两个不同的实体表示信息及其操作。\\

\subsection{面向对象(Object Oriented)}

面向对象是相对于面向过程来讲的,面向对象方法把相关的数据和方法组织为一个整体来看待,从更高的层次来进行系统建模,更贴近事物的自然运行模式。\\

在面向对象中,把构成问题的事物分解成各个对象,建立对象的目的不是为了完成一个步骤,而是为了描叙某个事物在整个解决问题的步骤中的行为。\\

Java、C++、Python等都是面向对象的编程语言,面向对象的优势在于只是用一个实体就能同时表示信息及其操作。\\

面向对象三大特性:

\begin{enumerate}
	\item 封装(encapsulation):数据和代码捆绑,避免外界干扰和不确定性访问。
	\item 继承(inheritance):让某种类型对象获得另一类型对象的属性和方法。
	\item 多态(polymorphism):同一事物表现出不同事物的能力。
\end{enumerate}

\newpage

\section{类与对象}

\subsection{类与对象}

类(class)表示同一类具有相同特征和行为的对象的集合,类定义了对象的属性和方法。\\

对象(object)是类的实例,对象拥有属性和方法。\\

类的设计需要使用关键字class,类名是一个标识符,遵循大驼峰命名法。类中可以包含属性和方法。其中,属性通过变量表示,又称实例变量;方法用于描述行为,又称实例方法。\\

通过关键字new进行对象的实例化,实例化对象会调用类中的构造函数完成。类是一种引用数据类型,对象的实例化在堆上开辟空间。\\

\mybox{类和对象}

\begin{lstlisting}[language=C++]
#include <iostream>
#include <string>

using namespace std;

class Person {
public:
    string name;
    int age;

    void eat() {
        cout << "吃饭" << endl;
    }

    void sleep() {
        cout << "睡觉" << endl;
    }
};

int main() {
    Person person;

    person.name = "小灰";
    person.age = 17;
    cout << "姓名:" << person.name << endl;
    cout << "年龄:" << person.age << endl;

    person.eat();
    person.sleep();
    return 0;
}
\end{lstlisting}

\begin{tcolorbox}
	\mybox{运行结果}
	\begin{verbatim}
姓名:小灰
年龄:17
吃饭
睡觉
	\end{verbatim}
\end{tcolorbox}

\newpage

\section{封装}

\subsection{封装(Encapsulation)}

封装是面向对象方法的重要原则,就是把对象的属性和方法结合为一个独立的整体,并尽可能隐藏对象的内部实现细节。\\

封装可以认为是一个保护屏障,防止该类的数据被外部类随意访问。要访问该类的数据,必须通过严格的接口控制。合适的封装可以让代码更容易理解和维护,也加强了程序的安全性。\\

实现封装的步骤:

\begin{enumerate}
	\item 修改属性的可见性来限制对属性的访问,一般限制为private。
	\item 对每个属性提供对外的公共方法访问,也就是提供一对setter / getter,用于对私有属性的访问。
\end{enumerate}

\vspace{0.5cm}

\subsection{访问权限}

属性和方法的访问权限一般分为3种:

\begin{enumerate}
	\item public:属性和方法在类的内部和外部都可以访问。
	\item private:属性和方法只能在类内访问。
	\item protected:属性和方法只能在类的内部和其派生类中访问。
\end{enumerate}

\vspace{0.5cm}

\subsection{this指针}

每一个对象都能通过this指针来访问自身的地址,this指针是所有成员方法的隐含参数,在成员方法内部可以用来指向调用对象。\\

在类中,属性的名字可以和局部变量的名字相同。此时,如果直接使用名字来访问,优先访问的是局部变量。因此,需要使用this指针来访问当前对象的属性。\\

当需要访问的属性与局部变量没有重名的时候,this可以省略。\\

\mybox{封装}

\begin{lstlisting}[language=C++]
#include <iostream>
#include <string>

using namespace std;

class Person {
public:
    void setName(string name) {
        this->name = name;
    }

    string getName() {
        return name;
    }

    void setAge(int age) {
        this->age = age;
    }

    int getAge() {
        return age;
    }

private:
    string name;
    int age;
};

int main() {
    Person person;

    person.setName("小灰");
    person.setAge(17);

    cout << "姓名:" << person.getName() << endl;
    cout << "年龄:" << person.getAge() << endl;
    return 0;
}
\end{lstlisting}

\begin{tcolorbox}
	\mybox{运行结果}
	\begin{verbatim}
姓名:小灰
年龄:17
	\end{verbatim}
\end{tcolorbox}

\newpage

\section{构造函数与析构函数}

\subsection{构造函数(Constructor)}

构造函数也是一个函数,用于实例化对象,在实例化对象的时候调用。一般情况下,使用构造函数是为了在实例化对象的同时,给一些属性进行初始化赋值。\\

构造函数和普通函数的区别:

\begin{enumerate}
	\item 构造函数的名字必须和类名一致。
	\item 构造函数没有返回值,返回值类型部分不写。
\end{enumerate}

如果一个类中没有构造函数,系统会自动提供一个public权限的无参构造函数以便实例化对象。如果一个类中已有构造函数,系统将不再提供任何默认的构造函数。\\

\mybox{构造函数}

\begin{lstlisting}[language=C++]
#include <iostream>
#include <string>

using namespace std;

class Person {
public:
    Person();
    Person(string name, int age);
    string toString();

private:
    string name;
    int age;
};

// 无参构造函数
Person::Person() {
    cout << "Person::Person()" << endl;
}

// 有参构造函数
Person::Person(string name, int age) {
    cout << "Person::Person(string, int)" << endl;
    this->name = name;
    this->age = age;
}

string Person::toString() {
    return "姓名:" + name + ",年龄:" + to_string(age);
}

int main() {
    Person p1;
    Person p2("小灰", 17);
    cout << p2.toString() << endl;
    return 0;
}
\end{lstlisting}

\begin{tcolorbox}
	\mybox{运行结果}
	\begin{verbatim}
Person::Person()
Person::Person(string, int)
姓名:小灰,年龄:17
	\end{verbatim}
\end{tcolorbox}

\vspace{0.5cm}

\subsection{初始化列表}

与其它函数不同,构造函数还可以有初始化列表。初始化列表以【:】开头,后跟一些列以逗号分割的初始化字段。\\

\mybox{初始化列表}

\begin{lstlisting}[language=C++]
#include <iostream>
#include <string>

using namespace std;

class Person {
public:
    Person(string name, int age);
    string toString();

private:
    string name;
    int age;
};

Person::Person(string name, int age) : name(name), age(age) {
    cout << "Person::Person(string, int)" << endl;
}

string Person::toString() {
    return "姓名:" + name + ",年龄:" + to_string(age);
}

int main() {
    Person person("小灰", 17);
    cout << person.toString() << endl;    
    return 0;
}
\end{lstlisting}

\begin{tcolorbox}
	\mybox{运行结果}
	\begin{verbatim}
Person::Person(string, int)
姓名:小灰,年龄:17
	\end{verbatim}
\end{tcolorbox}

有些时候初始化列表是不可或缺的,以下情况必须使用初始化列表:

\begin{enumerate}
	\item 常量成员:常量只能初始化不能赋值。

	\item 引用类型:引用必须在定义时初始化,且不能重新赋值。

	\item 没有默认构造函数的类类型:使用初始化列表可以不必调用默认构造函数来初始化,而是直接调用拷贝构造函数初始化。
\end{enumerate}

\vspace{0.5cm}

\subsection{析构函数(Destructor)}

析构函数与构造函数相反,当对象的生命周期结束时,会自动执行析构函数,用于做清理善后的事情。\\

析构函数的名称以【~】为前缀,后加类名称,它没有返回值和参数。\\

\mybox{析构函数}

\begin{lstlisting}[language=C++]
#include <iostream>
#include <string>

using namespace std;

class Person {
public:
    Person(string name, int age);
    ~Person();

private:
    string name;
    int age;
};

Person::Person(string name, int age) : name(name), age(age) {
    cout << "Person::Person(string, int)" << endl;
}

Person::~Person() {
    cout << "Person::~Person()" << endl;
}

int main() {
    Person p1("小灰", 17);
    Person *p2 = new Person("小白", 21);
    delete p2;
    return 0;
}
\end{lstlisting}

\begin{tcolorbox}
	\mybox{运行结果}
	\begin{verbatim}
Person::Person(string, int)
Person::Person(string, int)
Person::~Person()
Person::~Person()
	\end{verbatim}
\end{tcolorbox}

\vspace{0.5cm}

\subsection{拷贝构造函数(Copy Constructor)}

拷贝构造函数是构造函数的一种,它只有一个参数,参数类型为本类的引用。参数可以使const引用,也可以是非const引用,但是一般使用前者。\\

如果没有编写拷贝构造函数,编译器会自动生成一个默认的拷贝构造函数。\\

\mybox{拷贝构造函数}

\begin{lstlisting}[language=C++]
#include <iostream>
#include <string>

using namespace std;

class Person {
public:
    Person(const Person &p);
    Person(string name, int age);

private:
    string name;
    int age;
};

Person::Person(const Person &p) {
    cout << "Person::Person(const Person &)" << endl;
    this->name = p.name;
    this->age = p.age;
}

Person::Person(string name, int age) {
    cout << "Person::Person(string, int)" << endl;
    this->name = name;
    this->age = age;
}

int main() {
    Person p1("小灰", 17);
    Person p2(p1);
    Person p3 = p1;
    return 0;
}
\end{lstlisting}

\begin{tcolorbox}
	\mybox{运行结果}
	\begin{verbatim}
Person::Person(string, int)
Person::Person(const Person &)
Person::Person(const Person &)
	\end{verbatim}
\end{tcolorbox}

拷贝构造函数会在三种情况下被调用:

\begin{enumerate}
	\item 用一个对象去初始化同类的另一个对象。
	\item 函数参数是类的对象。
	\item 函数的返回值是类的对象。
\end{enumerate}

\vspace{0.5cm}

\subsection{浅拷贝 / 深拷贝}

当使用浅拷贝(shallow copy)时,仅仅是拷贝指针字面值,如果原来的对象调用析构函数释放掉指针所指向的数据,则会产生空悬指针(dangling pointer),因为所指向的内存空间已经被释放了。\\

\mybox{浅拷贝}

\begin{lstlisting}[language=C++]
#include <iostream>

using namespace std;

class User {
public:
    User();
    ~User();
    void printDataAddress();

private:
    int *data;
};

User::User() {
    this->data = new int;
}

User::~User() {
    delete data;
    data = nullptr;
}

void User::printDataAddress() {
    cout << data << endl;
}

int main() {
    User user1;
    user1.printDataAddress();
    User user2(user1);
    user2.printDataAddress();  
    return 0;
}
\end{lstlisting}

\begin{tcolorbox}
	\mybox{运行结果}
	\begin{verbatim}
0x26c2b90
0x26c2b90
	\end{verbatim}
\end{tcolorbox}

深拷贝(deep copy)可以解决浅拷贝出现的问题,通过定义一个拷贝构造函数,当被拷贝对象存在动态分配的存储空间时,需要先动态申请一块存储空间,然后逐字节拷贝内容。\\

\mybox{深拷贝}

\begin{lstlisting}[language=C++]
#include <iostream>

using namespace std;

class User {
public:
    User();
    User(const User& user);
    ~User();
    void printDataAddress();

private:
    int *data;
};

User::User() {
    this->data = new int;
}

User::User(const User& user) {
    this->data = new int;
    *(this->data) = *(user.data);
}

User::~User() {
    delete data;
    data = nullptr;
}

void User::printDataAddress() {
    cout << data << endl;
}

int main() {
    User user1;
    user1.printDataAddress();
    User user2(user1);
    user2.printDataAddress();  
    return 0;
}
\end{lstlisting}

\begin{tcolorbox}
	\mybox{运行结果}
	\begin{verbatim}
0x6b17b0
0x6b17d0
	\end{verbatim}
\end{tcolorbox}

\newpage

\section{静态成员}

\subsection{静态成员}

类的静态成员在编译时创建并初始化,在该类的任何对象建立之前就已经存在。静态成员不属于任何对象,并且在类中只有一份,为所有此类对象共享。\\

在静态成员函数的实现中不能直接引用类中的非静态成员,但可以引用类中的静态成员。如果静态成员函数中要引用非静态成员时,需要通过对象来引用。\\

\mybox{静态成员}

\begin{lstlisting}[language=C++]
#include <iostream>
#include <string>

using namespace std;

class User {
public:
    User(int id, string name) : id(id), name(name) {
        totalUsers++;
    }

    static int getTotalUsers() {
        return totalUsers;
    }

private:
    static int totalUsers;
    int id;
    string name;
};

int User::totalUsers = 0;       // 初始用户数量

int main() {
    cout << User::getTotalUsers() << endl;

    for(int i = 0; i < 10; i++) {
        User user(i, "User-" + to_string(i));
    }

    cout << User::getTotalUsers() << endl;
    return 0;
}
\end{lstlisting}

\begin{tcolorbox}
	\mybox{运行结果}
	\begin{verbatim}
0
10
	\end{verbatim}
\end{tcolorbox}

\newpage

\section{友元}

\subsection{友元函数}

封装使得类的数据对外隐藏,但是有些函数不是类的一部分,却又需要频繁访问类的数据成员,这时可以将这些函数定义为该类的友元函数。一个函数可以是多个类的友元函数,只需要在各个类中分别声明。除了友元函数,还有友元类。\\

友元(friend)的作用是提高程序的运行效率,减少了类型检查和安全性检查等需要的时间开销,但它破坏了类的封装性和隐藏性,使得非成员函数可以访问类的私有成员。\\

友元函数是可以直接访问类的私有成员的非成员函数。它是定义在类外的普通函数,它不属于任何类,但需要在类的定义中加以声明。

\vspace{-0.5cm}

\begin{lstlisting}[language=C++]
friend ret_type func_name([param_list]);
\end{lstlisting}

\vspace{0.5cm}

\mybox{友元函数}

\begin{lstlisting}[language=C++]
#include <iostream>
#include <cmath>

using namespace std;

class Coordinate {
public:
    Coordinate(double x, double y) : x(x), y(y) {};

    friend double distance(Coordinate &c1, Coordinate &c2);

private:
    double x;
    double y;
};

double distance(Coordinate &c1, Coordinate &c2) {
    double deltaX = c1.x - c2.x;
    double deltaY = c1.y - c2.y;
    return sqrt(deltaX * deltaX + deltaY * deltaY);
}

int main() {
    Coordinate c1(3, 5);
    Coordinate c2(4, 6);
    cout << distance(c1, c2) << endl;
    return 0;
}
\end{lstlisting}

\begin{tcolorbox}
	\mybox{运行结果}
	\begin{verbatim}
1.41421
	\end{verbatim}
\end{tcolorbox}

\vspace{0.5cm}

\subsection{友元类}

友元类的所有成员函数都是另一个类的友元函数,可以访问另一个类中的隐藏信息。当一个类想要存取另一个类的私有成员时,可以将该类声明为另一类的友元类。

\vspace{-0.5cm}

\begin{lstlisting}[language=C++]
friend class class_name;
\end{lstlisting}

友元有以下需要注意的地方:

\begin{enumerate}
	\item 友元关系不能被继承。
	\item 友元关系是单向的,不具有交换性。如果A是B的友元,B不一定是A的友元。
	\item 友元关系不具有传递性。如果A是B的友元,C是A的友元,那么C不一定是B的友元。
\end{enumerate}

\newpage

\section{运算符重载}

\subsection{运算符重载}

C++中预定义的运算符的操作对象只能是基本数据类型,但实际上对于许多用户自定义类型(例如类),也需要类似的运算操作。这时就必须在C++中重新定义这些运算符,赋予已有运算符新的功能,使它能够用于特定类型执行特定的操作。运算符重载的实质是函数重载,它提供了可扩展性。\\

运算符重载是通过创建运算符函数实现的,运算符函数定义了重载的运算符将要进行的操作。运算符函数的定义与其它函数的定义类似,惟一的区别是运算符函数的函数名是由关键字operator和要重载的运算符符号构成。

\vspace{-0.5cm}

\begin{lstlisting}[language=C++]
ret_type operator op([param_list]) {
    // code
}
\end{lstlisting}

运算符重载需要遵循以下规则:

\begin{enumerate}
	\item 除了【.】、【->】、【sizeof】、【?:】和【\#】,其它运算符都可以重载。

	\item 重载后的运算符不能改变优先级和结合性,也不能概念运算符的操作数个数及语法结构。

	\item 运算符重载是针对新类型数据对实际需要的改造,重载后的运算符应当与原有功能相类似。
\end{enumerate}

\vspace{0.5cm}

\subsection{二元运算符重载}

二元运算符需要两个操作数,例如【+】、【-】、【*】、【/】等。\\

\mybox{二元运算符重载}

\begin{lstlisting}[language=C++]
#include <iostream>
#include <string>

using namespace std;

class Complex {
public:
    Complex(int real, int imaginary);
    string getNumber();
    Complex operator+(const Complex& c);

private:
    int real;
    int imaginary;
};

Complex::Complex(int real = 0, int imaginary = 0)
    : real(real), imaginary(imaginary) {}

string Complex::getNumber() {
    return to_string(real) + "+" + to_string(imaginary) + "i";
}

Complex Complex::operator+(const Complex& c) {
    Complex complex;
    complex.real = this->real + c.real;
    complex.imaginary = this->imaginary + c.imaginary;
    return complex;
}

int main() {
    Complex c1(1, 2);
    Complex c2(8, 1);
    Complex result = c1 + c2;
    cout << result.getNumber() << endl;
    return 0;
}
\end{lstlisting}

\begin{tcolorbox}
	\mybox{运行结果}
	\begin{verbatim}
9+3i
	\end{verbatim}
\end{tcolorbox}

\vspace{0.5cm}

\subsection{一元运算符重载}

一元运算符只对一个操作数操作,例如【++】、【--】、【-】、【!】等。\\

\mybox{一元运算符重载}

\begin{lstlisting}[language=C++]
#include <iostream>
#include <string>
#include <iomanip>

using namespace std;

class Time {
public:
    Time(int hour, int minute, int second);
    void display();
    Time operator++();      // 前置++
    Time operator++(int);   // 后置++

private:
    int hour;
    int minute;
    int second;
};

Time::Time(int hour, int minute, int second) 
    : hour(hour), minute(minute), second(second) {}

void Time::display() {
    cout << setfill('0')
         << setw(2) << hour << ":"
         << setw(2) << minute << ":"
         << setw(2) << second << endl;
}

// 前置++
Time Time::operator++() {
    second++;
    if(second == 60) {
        second %= 60;
        minute++;
        if(minute == 60) {
            minute %= 60;
            hour++;
            if(hour == 24) {
                hour = 0;
            }
        }
    }
    return Time(hour, minute, second);
}

// 后置++
Time Time::operator++(int) {
    // 保存原始值
    Time time(hour, minute, second);
    second++;
    if(second == 60) {
        second %= 60;
        minute++;
        if(minute == 60) {
            minute %= 60;
            hour++;
            if(hour == 24) {
                hour = 0;
            }
        }
    }
    return time;    // 返回原始值
}

int main() {
    Time time(9, 21, 58);
    time.display();

    ++time;
    time.display();

    time++;
    time.display();
    return 0;
}
\end{lstlisting}

\begin{tcolorbox}
	\mybox{运行结果}
	\begin{verbatim}
09:21:58
09:21:59
09:22:00
	\end{verbatim}
\end{tcolorbox}

\vspace{0.5cm}

\subsection{输入输出运算符重载}

C++使用流提取运算符【>>】和流插入运算符【<<】进行输入输出,通过运算符重载可以对自定义对象进行输入输出操作。通过把输入输出运算符重载函数声明为类的友元,可以直接调用函数而无需创建对象。\\

\mybox{输入输出运算符重载}

\begin{lstlisting}[language=C++]
#include <iostream>
#include <string>
using namespace std;

class User {
public:
    User(int id, string name);
    friend ostream& operator<<(
        ostream& out,
        const User& user);
    friend istream& operator>>(istream& in, User& user);

private:
    int id;
    string name;
};

User::User(int id = 0, string name = "")
    : id(id), name(name) {}

ostream& operator<<(ostream& out, const User& user) {
    out << "ID: " << to_string(user.id) << ", "
        << "name: " << user.name;
    return out;
}

istream& operator>>(istream& in, User& user) {
    cout << "Enter user ID: ";
    in >> user.id;
    cout << "Enter user name: ";
    in >> user.name;
    return in;
}

int main() {
    User user;
    cin >> user;
    cout << user;
    return 0;
}
\end{lstlisting}

\begin{tcolorbox}
	\mybox{运行结果}
	\begin{verbatim}
Enter user ID: 1
Enter user name: Terry
ID: 1, name: Terry
	\end{verbatim}
\end{tcolorbox}

\newpage