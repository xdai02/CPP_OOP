\chapter{I/O库}

\section{标准I/O}

\subsection{标准I/O}

C++不直接处理输入输出,而是通过标准库中的一组类来处理I/O。输入流istream提供输入,输出流ostream提供输出。\\

标准输入输出有以下特点:

\begin{enumerate}
	\item cin是istream的对象,从标准输入读取数据。
	\item cout是ostream的对象,向标准输出写数据。
	\item cerr是ostream的对象,用于输出错误信息,写到标准错误。
	\item 【>>】运算符从istream对象读取输入。
	\item 【<<】运算符从ostream对象写输出。
	\item getline()从给定的`istream`读取一行数据,存入string对象。
\end{enumerate}

\vspace{0.5cm}

\subsection{I/O格式化}

每个iostream对象维护一个格式状态来控制I/O的细节,如进制、精度、宽度等。操纵符改变流的格式状态时,通常改变后的状态对所有后续I/O都生效。\\

标准库定义了一组操纵符用来修改流的格式状态:

\begin{table}[H]
	\centering
	\setlength{\tabcolsep}{5mm}{
		\begin{tabular}{|c|l|}
			\hline
			\textbf{操纵符} & \textbf{功能}                           \\
			\hline
			boolalpha       & 将true和false输出为字符串               \\
			\hline
			noboolalpha     & 将true和false输出为1和0                 \\
			\hline
			showbase        & 对整型值输出表示进制的前缀              \\
			\hline
			noshowbase      & 不生成表示进制的前缀                    \\
			\hline
			showpoint       & 浮点数总是显示小数点                    \\
			\hline
			noshowpoint     & 只有浮点数包含小数部分才显示小数点      \\
			\hline
			showpos         & 非负数显示【+】                         \\
			\hline
			noshowpos       & 非负数不显示【+】                       \\
			\hline
			uppercase       & 在十六进制中打印0X,在科学计数法中打印E \\
			\hline
			nouppercase     & 在十六进制中打印0x,在科学计数法中打印e \\
			\hline
			dec             & 整型显示为十进制                        \\
			\hline
			hex             & 整型显示为十六进制                      \\
			\hline
			oct             & 整型显示为八进制                        \\
			\hline
			left            & 在值的左侧添加填充字符                  \\
			\hline
			right           & 在值的右侧添加填充字符                  \\
			\hline
			internal        & 在符号和值之间添加填充字符              \\
			\hline
			fixed           & 浮点数显示为定点十进制                  \\
			\hline
			scientific      & 浮点数显示为科学计数法                  \\
			\hline
			unitbuf         & 每次输出操作后刷新缓冲区                \\
			\hline
			nounitbuf       & 恢复正常的缓冲区刷新方式                \\
			\hline
			skipws          & 输入运算符跳过空白符                    \\
			\hline
			noskipws        & 输入运算符不跳过空白符                  \\
			\hline
			flush           & 刷新ostream缓冲区                       \\
			\hline
			ends            & 插入空字符,然后刷新ostream缓冲区       \\
			\hline
			endl            & 插入换行符,然后刷新ostream缓冲区       \\
			\hline
			setfill(ch)     & 用ch填充空白                            \\
			\hline
			setprecision(n) & 将浮点精度设置为n                       \\
			\hline
			setw(n)         & 读或写值的宽度为n个字符                 \\
			\hline
			setbase(n)      & 将整数输出为n进制                       \\
			\hline
		\end{tabular}
	}
	\caption{操纵符}
\end{table}

\mybox{格式化输出}

\begin{lstlisting}[language=C++]
#include <iostream>
#include <cmath>
#include <iomanip>

using namespace std;

int main() {
    cout << "布尔:";
    cout << boolalpha << true << " " << false << endl;
    cout << "------------------------------" << endl;
    
    cout << "十进制:";
    cout << dec << 20 << " " << 1024 << endl;
    
    cout << "十六进制:";
    cout << showbase << hex 
        << 20 << " "<< 1024
        << noshowbase << endl;
    
    cout << "八进制:";
    cout << oct << 20 << " " << 1024 << dec << endl;
    cout << "------------------------------" << endl;
    
    cout << "科学计数法:";
    cout << scientific
        << 100 * sqrt(2)
        << defaultfloat << endl;
    cout << "------------------------------" << endl;
    
    cout << "默认输出浮点数:";
    cout << 10.0 << endl;
    
    cout << "浮点数打印小数点:";
    cout << showpoint << 10.0 << noshowpoint << endl;
    cout << "------------------------------" << endl;
    
    cout << "精度:";
    cout << setprecision(3) << fixed << sqrt(2) << endl;
    cout << "------------------------------" << endl;
    
    cout << "宽度填充:";
    cout << setfill('0') << setw(4) << 2021 << "/"
         << setw(2) << 9 << "/"
         << setw(2) << 2 << endl;
    return 0;
}
\end{lstlisting}

\begin{tcolorbox}
	\mybox{运行结果}
	\begin{verbatim}
布尔:true false
------------------------------
十进制:20 1024
十六进制:0x14 0x400
八进制:24 2000
------------------------------
科学计数法:1.414214e+02
------------------------------
默认输出浮点数:10
浮点数打印小数点:10.0000
------------------------------
精度:1.414
------------------------------
宽度填充:2021/09/02
	\end{verbatim}
\end{tcolorbox}

\newpage

\section{文件I/O}

\subsection{文件I/O}

程序不仅要从控制台进行I/O,还需要读写文件和字符串。\\

标准库的I/O类型在3个头文件中:

\begin{enumerate}
	\item <iostream>定义了读写流的基本类型。
	\item <fstream>定义了读写文件的类型。
	\item <sstream>定义了读写string对象的类型。
\end{enumerate}

<fstream>中定义了3个I/O类来读写文件:

\begin{enumerate}
	\item ifstream从给定文件读数据。
	\item ofstream向给定文件写数据。
	\item fstream可读写文件。
\end{enumerate}

\vspace{0.5cm}

\subsection{文件打开模式}

每个流都有一个关联的文件模式,在打开文件时可以指定文件模式。

\begin{table}[H]
	\centering
	\setlength{\tabcolsep}{5mm}{
		\begin{tabular}{|c|l|}
			\hline
			\textbf{打开模式} & \textbf{作用}                                   \\
			\hline
			ios::in           & 以读方式打开                                    \\
			\hline
			ios::out          & 以写方式打开                                    \\
			\hline
			ios::app          & 以追加方式打开                                  \\
			\hline
			ios::ate          & 打开文件定位到文件末尾                          \\
			\hline
			ios::trunc        & 如果文件存在,其内容将被截断,即把文件长度设为0 \\
			\hline
		\end{tabular}
	}
	\caption{文件打开模式}
\end{table}

\mybox{文件I/O}

\begin{lstlisting}[language=C++]
#include <iostream>
#include <fstream>

using namespace std;

int main() {
    string name;
    int id;
    cout << "Enter name: ";
    cin >> name;
    cout << "Enter id: ";
    cin >> id;
    
    ofstream out("info.txt");
    out << name << " " << id << endl;
    out.close();
    
    ifstream in("info.txt");
    in >> name >> id;
    in.close();
    
    cout << "name = " << name << ", id = " << id << endl;
    return 0;
}
\end{lstlisting}

\begin{tcolorbox}
	\mybox{运行结果}
	\begin{verbatim}
Enter name: Terry
Enter id: 979489
name = Terry, id = 979489
	\end{verbatim}
\end{tcolorbox}

\newpage

\section{string流}

\subsection{string流}

<sstream>定义了3个类来支持内存IO:

\begin{enumerate}
	\item istringstream从string读数据。
	\item ostringstream向string写数据。
	\item stringstream可读写string。
\end{enumerate}

\mybox{string流}

\begin{lstlisting}[language=C++]
#include <iostream>
#include <sstream>
#include <string>

using namespace std;

int main() {
    string line;
    cout << "convert a string to Python list format: ";
    getline(cin, line);

    ostringstream out;
    istringstream in(line);
    string token;

    out << "[";
    while(in >> token) {
        out << token << ", ";
    }
    out << "\b\b]";
    cout << out.str() << endl;
    return 0;
}
\end{lstlisting}

\begin{tcolorbox}
	\mybox{运行结果}
	\begin{verbatim}
convert a string to Python list format: This is a test
[This, is, a, test]
	\end{verbatim}
\end{tcolorbox}

\newpage